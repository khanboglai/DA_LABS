\section{Выводы}
% {\itshape Здесь Вы пишите то, чему научились на лабораторной на самом деле, что узнали нового, где это может пригодиться и т.д. Мне важно, какие именно Вы сделали выводы из лабораторной.}

Выполнив девятую лабораторную работу по курсу \enquote{Дискретный анализ}, 
я ознакомился с графами и изучил способы представления графов в компьютере. Также я познакомился с 
базовыми алгоритмами поиска кратчайших путей в графе (алгоритм Дейкстры и алгоритм Беллмана-Форда).

Алгоритм Дейкстры, основанный на жадном подходе, эффективно находит кратчайшие пути от одной вершины до всех 
остальных в графах с неотрицательными весами рёбер. Его временная сложность составляет $O((V + E) log V)$ при 
использовании очереди с приоритетом, что делает его подходящим для разреженных графов. 
Однако алгоритм не может корректно обрабатывать графы с отрицательными весами.

С другой стороны, алгоритм Беллмана-Форда, хотя и менее эффективный с временной сложностью $O(V * E)$, 
способен работать с графами, содержащими отрицательные веса, и даже обнаруживать отрицательные циклы. 
Это делает его более универсальным инструментом для решения задач, где возможны такие условия.

Графы широко используются в нашей повседневной жизни. Например, построение кратчайшего маршрута до дома.
Или выбор места строительства нового торгового центра, нужно построить в быстро-доступном для всех месте.

\pagebreak