\CWHeader{Лабораторная работа \textnumero 9}

\CWProblem{
    Разработать программу на языке C или C++, реализующую алгоритм согласно условию задачи.


{\bfseries Вариант 4: Поиск кратчайшего пути между парой вершин алгоритмом Дейкстры} 

Задан взвешенный неориентированный граф, состоящий из n вершин и m ребер. 
Вершины пронумерованы целыми числами от $1$ до $n$. Необходимо найти длину кратчайшего пути из вершины с номером 
$start$ в вершину с номером $finish$ при помощи алгоритма Дейкстры. Длина пути равна сумме весов ребер на этом пути. 
Граф не содержит петель и кратных ребер.  

{\bfseries Формат ввода:}

В первой строке заданы $1 \leq n \leq 10^5$ и $1 \leq m \leq 10^5$, $1 \leq start \leq n$ и $1 \leq finish \leq n$. 
В следующих m строках записаны ребра. Каждая строка содержит три числа – номера вершин, 
соединенных ребром, и вес данного ребра. Вес ребра – целое число от $0$ до $10^9$.

{\bfseries Формат вывода:}

Необходимо вывести одно число – длину кратчайшего пути между указанными вершинами. 
Если пути между указанными вершинами не существует, следует вывести строку No solution. 
}

\pagebreak