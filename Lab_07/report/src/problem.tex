\CWHeader{Лабораторная работа \textnumero 7}

\CWProblem{
    Используя метод динамического программирования разработать алгоритм решения задачи, 
    определяемой своим вариантом: оценить время работы алгоритма и объем затрачиваемой памяти. Перед выволнением 
    задания необходимо обосновать пременимость метода динамического программирования.

    Разработать программу на языке С++ или С, реализующую построенный алгоритм. Формат данных описан 
    в варианте задания.


{\bfseries Вариант 4:} Имеется натуральное число $n$. За один ход с ним можно произвести следующие действия:

\begin{itemize}
    \item Вычесть единицу
    \item Разделить на два
    \item Разделить на три
\end{itemize}

При этом стоимость каждой операции – текущее значение $n$. 
Стоимость преобразования - суммарная стоимость всех операций в преобразовании. 
Вам необходимо с помощью последовательностей указанных операций преобразовать число $n$ в единицу таким образом, 
чтобы стоимость преобразования была наименьшей. Делить можно только нацело. 

}
\pagebreak