\section{Тест производительности}
% {\itshape Тут Вы описываете собственно тест производительности, сравнение Вашей реализации с уже существующими и т.д.}
Сравним алгоритм с наивным алгоритмом. Тесты состоят из чисел: 82, 100, 200, 500. 

\begin{alltt}
alex@wega:~/$ ./a.out 
82
Naive: 0.534 ms
DP: 0.107 ms
alex@wega:~/$ ./a.out 
100
Naive: 0.742 ms
DP: 0.079 ms
alex@wega:~/$ ./a.out 
200
Naive: 9.439 ms
DP: 0.038 ms
alex@wega:~/$ ./a.out 
500
Naive: 641.927 ms
DP: 0.056 ms
\end{alltt}

Можно заметить, что динамическое программирование работает гораздо быстрее наивного алгоритма, потому что в наивном алгоритме 
стоимость вычисляется на каждом шаге, в то время как в динамическом программировании используется результат предыдущего шага.

\pagebreak
