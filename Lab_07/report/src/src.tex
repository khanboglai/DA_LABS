\section{Описание}

Согласно \cite{Kormen}: \enquote{Динамическое программировние позволяет решать задачи, комбинируя решения 
вспомогательный задач. Каждая вспомогательная задача решается только
один раз. Это позволяет избежать одних и тех же повторных вычислений каждый раз, 
когда встречается данная подзадача.}

Динамическое программирование, как правило, применяется к задачам оптимизации (optimization problems). 
В таких задачах возможно наличие многих решений. Каждому варианту решения можно сопоставить какое-то значение, 
и нам нужно найти среди них решение с оптимальным (минимальным или максимальным) значением.

Процесс разработки алгоритмов динамического программирования можно разбить на четыре перечисленных ниже этапа:

\begin{enumerate}
    \item Описание структуры оптимального решения.
    \item Рекурсивное определение значения, соответствующего оптимальному решению.
    \item Вычисление значения, соответствующего оптимальному решению, с помощью метода восходящего анализа.
    \item Составление оптимального решения на основе информации, полученной на предыдущих этапах.
\end{enumerate}


Сохранение результатов выполнения функций для предотвращения повторных вычислений называется \textbf{мемоизацией}.
Перед вызовом функции производится проверка: вызывалась функция или нет. Если функция вызывалась, 
то нужно взять результат ее вызова.

Задачей, где можно применить мемоизацию, является задача о нахождении числа Фибоначчи под номером $n$.

Также в динамическом программировании используются методы восходящего и нисходящего анализов для решения задачи.
В восходящем методе сначала решаются простые задачи, а после более сложные. В нисходящем наоборот, сначала сложные, а
после простые. 


\pagebreak

\section{Исходный код}
% Здесь должно быть подробное описание программы и основные этапы написания кода.

Нужно посчитать стоимость преобразования исходного числа в 1. Чтобы это сделать, нужно разработать алгоритм 
динамического программирования.

Есть три операции над числом: вычесть 1, разделить на 2, разделить на 3. Для составления алгоритма решения задачи 
используем метод восходящего анализа, вычислим стоимость преобразования для чисел, стоящих перед исходным числом.

Например, стоимость преобразования из числа 1 в 1 равно 0. Для числа 2 стоимость равна 2 и тд.

Создадим векстор длины $n + 1$, чтобы работать с числами от 1 до $n$. Далее для каждого числа, начиная с 2, вычислим
стоимость преобразования в 1. Причем будем проверять, какую операцию использовать дешевле, используя предыдущие результаты.

Так мы сможем найти минимальную стоимость преобразования.


Также создадим вектор для сохранения операций, которые мы производим над числами. В функции $MakeSequence$ 
восстанавливаем последовательность операций, которые мы проделали для преобразования.

Для удобства создали пару $TPair$, чтобы вернуть из функции минимальную стоимость и послеловательность операций.
% В этом случае структуры или классы должны быть полностью приведены в листинге (без реализации методов).

\begin{lstlisting}[language=C]
#include <iostream>
#include <vector>
#include <string>
#include <algorithm>


using TPair = std::pair<int, std::vector<std::string>>;


std::vector<std::string> MakeSequence(int n, std::vector<std::string> &opers) {
    std::vector<std::string> seq;

    int cur_num = n;

    while (cur_num > 1) {
        seq.push_back(opers[cur_num]);

        if (opers[cur_num] == "-1") {
            cur_num -= 1;
        } else if (opers[cur_num] == "/2") {
            cur_num /= 2;
        } else if (opers[cur_num] == "/3") {
            cur_num /= 3;
        }
    }
    return seq;
}


TPair MinCost(int n) {
    std::vector<long long> dp(n + 1, 0);
    std::vector<std::string> opers(n + 1, "");

    for (int i = 2; i <= n; i++) {

        dp[i] = dp[i - 1] + i;
        opers[i] = "-1";

        if (i % 2 == 0 && dp[i] > dp[i / 2] + i) {
            dp[i] = dp[i / 2] + i;
            opers[i] = "/2";
        }

        if (i % 3 == 0 && dp[i] > dp[i / 3] + i) {
            dp[i] = dp[i / 3] + i;
            opers[i] = "/3";
        }
    }

    std::vector<std::string> seq = MakeSequence(n, opers);

    return std::make_pair(dp[n], seq);
}


int main() {
    int n;
    std::cin >> n;
    TPair res = MinCost(n);
    std::cout << res.first << std::endl;
    
    for (const std::string & str : res.second) {
        std::cout << str << " ";
    }
    
    std::cout << std::endl;
    return 0;
}
\end{lstlisting}


\pagebreak

\section{Консоль}
\begin{alltt}
alex@wega:~/$ ./main 
12
18
/3 /2 -1 
alex@wega:~/$ ./main 
82
202
-1 /3 /3 /3 /3 
alex@wega:~/$ ./main 
100
213
/2 /2 -1 /3 /2 /2 -1 
alex@wega:~/$ ./main 
3421
8839
-1 /3 /3 /2 -1 /3 /3 /3 -1 /3 -1 
alex@wega:~/$
\end{alltt}
\pagebreak
