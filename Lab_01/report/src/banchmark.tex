\section{Тест производительности}
% {\itshape Тут Вы описываете собственно тест производительности, сравнение Вашей реализации с уже существующими и т.д.}


Тест производительности представляет из себя следующее: будем сравнивать время выполнения нашей поразрядной сортировки и 
встроенной сортировки.

\begin{alltt}
alex@wega:~/da_labs/Lab_01$ make
g++ -std=c++17 -pedantic -Wall main.cpp -o lab1
g++ -std=c++17 -pedantic -Wall benchmark.cpp -o bench1
alex@wega:~/da_labs/Lab_01$ ./bench1 < tests/01.t
Count lines is: 100
Radix_sort time: 0.243 ms
Stable_sort time: 0.24 ms
Difference: 0.987654
alex@wega:~/da_labs/Lab_01$ ./bench1 < tests/01.t
Count lines is: 1000
Radix_sort time: 0.677 ms
Stable_sort time: 1.118 ms
Difference: 1.6514
alex@wega:~/da_labs/Lab_01$ ./bench1 < tests/01.t
Count lines is: 10000
Radix_sort time: 7.36 ms
Stable_sort time: 16.426 ms
Difference: 2.23179
alex@wega:~/da_labs/Lab_01$ ./bench1 < tests/01.t
Count lines is: 50000
Radix_sort time: 52.646 ms
Stable_sort time: 100.743 ms
Difference: 1.91359
alex@wega:~/da_labs/Lab_01$
\end{alltt}

Можем заметить, что поразрядная сортировка примерно в 2 раза быстрее, это можно объяснить тем, что сложность поразрядной сортировки
$O(n * k)$ (n - кол-во элементов в последовательности, а k - количество значащих разрядов в максимальном элементе), а у встроенной сортировки сложность $O(n*log n)$. 

\pagebreak
