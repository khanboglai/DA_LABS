\section{Описание}

Согласно \cite{Kormen}: \enquote{Жадный алгоритм позволяет получить оптимальное решение задачи путем
осуществления ряда выборов. В каждой точке принятия решения в алгоритме делается выбор, 
который в данный момент выглядит самым лучшим. Эта эвристическая стратегия не всегда дает оптимальное решение, 
но все же решение может оказаться и оптимальным.}

{\bfseries Свойство жадных алгоритмов:} глобальное оптимальное решение можно получить,
делая локальный оптимальный (жадный) выбор.

{\bfseries Процесс построения жадных алгоритмов:}

\begin{enumerate}
    \item Привести задачу оптимизации к виду, когда после сделанного выбора остается решить только одну подзадачу.
    \item Доказать, что всегда существует такое оптимальное решение исходной задачи, 
    которое можно получить путем жадного выбора, так что такой выбор всегда допустим.
    \item Показать, что после жадного выбора остается подзадача, обладающая тем
    свойством, что объединение оптимального решения подзадачи со сделанным
    жадным выбором приводит к оптимальному решению исходной задачи.
\end{enumerate}


Для решения задачи нам потребуется вспомнить теорию из курса Линейной алгебры. А точнее, что такое линейная независимость строк и 
столбцов матрицы, как приводить матрицу к ступенчатому виду и находить ее ранг.

Для нахождения коэффициентов в формуле для расчета воздейсвтвия добавки, необходимо составить матрицу из значений, поступающих на
ввод программе. Нужно найти $N$ коэффициентов, это говорит нам о том, что нужно найти ранг матрицы и проверить, равен ли он числу столбцов.

Ранг --- это количество не нулевых строк матрицы. Если ранг не будет равен числу столбцов, значит найти коэффициенты нельзя.

Ранг можно найти путём приведения матрицы к ступенчатому виду. Еще одно определение ранга матрицы: это максимальное число ее линейно 
независимых строк(столбцов). Самое главное --- ранг не может быть больше, чем $min(m, n)$.

Если расположить строки матрицы по возрастанию цены, сверху вниз, то можно найти набор добавок с наименьшей ценой.

Данный подход и будет являться жадным алгоритмом, так как ненулевые строки, составляющие ранг матрицы, будут иметь наименьшую цену.
И вместо перебора всевозможных линейно независимых строк с наименьшей ценой, будет рассмотрено $N$ первые линейно независимые строки.

\pagebreak

\section{Исходный код}
% Здесь должно быть подробное описание программы и основные этапы написания кода.

Для реализации алгоритма напишем функцию $IsLinearIndependent$ для поиска $N$ линейно незавимых строк. Сначала матрица приводится к ступенчатому виду
при помощи метода Гаусса, далее проверяется, равен ли ранг матрицы количеству столбцов и возвращается результат сравнения.

Чтобы не создавать дополнительных переменных и структур, увеличим количетсво столбцов матрицы для сохранения цену мешка добавок
и номер строки во входный данных.

Чтобы расположить строки матрицы по возрастанию цены мешка добавки, отсортируем строки по цене, предворительно, 
написав функцию $ComporatorCost$ для сравения векторов по предпоследнему элементу.

Если в мы смогли найти $N$ линейно независимых строк в матрице, то нужно сохранить их исходные номера в вектор и отсортировать его.

Далее выводим отсортированный вектор.
% В этом случае структуры или классы должны быть полностью приведены в листинге (без реализации методов).

\begin{lstlisting}[language=C]
#include <bits/stdc++.h>


bool IsLinearIndependent(std::vector<std::vector<double>> &matr) {
    int rows = matr.size();
    int cols = matr[0].size() - 2;
    int rank = 0;

    for (int col = 0; col < cols; col++) {
        int current_row = -1;
        
        for (int row = rank; row < rows; row++) {
            if (matr[row][col] != 0) {
                current_row = row;
                break;
            }
        }

        if (current_row == -1) {
            continue;
        }

        std::swap(matr[rank], matr[current_row]);

        for (int row = 0; row < rows; row++) {
            if (row != rank) {
                double factor = matr[row][col] / matr[rank][col];

                for (int j = col; j < cols; j++) {
                    matr[row][j] -= factor * matr[rank][j];
                }
            }
        }
        rank++;
    }
    return rank == cols;
}


bool ComporatorCost(const std::vector<double> &a, const std::vector<double> &b) {
    return a[a.size() - 2] < b[b.size() - 2];
}


int main() {
    int m, n;
    std::cin >> m >> n;

    std::vector<std::vector<double>> matrix(m, std::vector<double>(n + 2));

    for (int i = 0; i < m; i++) {
        for (int j = 0; j < n + 1; j++) {
            std::cin >> matrix[i][j];
        }
        matrix[i][n + 1] = i;
    }

    std::sort(matrix.begin(), matrix.end(), ComporatorCost);

    if (IsLinearIndependent(matrix)) {
        std::vector<int> res;
        for (int i = 0; i < n; i++) {
            res.push_back(matrix[i].back() + 1);
        }
    
        std::sort(res.begin(), res.end());
        for (const int &elem : res) {
            std::cout << elem << " ";
        }
        std::cout << std::endl;
    } else {
        std::cout << -1 << std::endl;
    }
    return 0;
}   
\end{lstlisting}


\pagebreak

\section{Консоль}
\begin{alltt}
alex@wega:~/$ cat tests/01.t 
2 2
45 17 21
50 14 38
alex@wega:~/$ ./main < tests/01.t 
1 2 
alex@wega:~/$ cat tests/01.t 
3 2
34 4 48
50 19 33
5 23 50
alex@wega:~/$ ./main < tests/01.t 
1 2 
alex@wega:~/$ cat tests/01.t 
7 6
35 39 0 11 0 39 23
4 9 36 12 31 38 5
23 43 41 28 27 23 2
26 9 49 41 49 26 32
29 1 42 18 32 24 17
45 0 49 12 48 11 35
22 1 3 34 38 6 45
alex@wega:~/$ ./main < tests/01.t 
1 2 3 4 5 6 
alex@wega:~/$ cat tests/01.t 
14 14
29 43 3 21 21 32 17 14 3 37 21 4 39 40 27
41 13 2 20 21 3 37 22 47 9 50 10 42 29 37
16 38 47 31 39 23 30 32 30 25 1 0 41 17 31
44 32 25 34 17 1 40 21 8 0 45 29 46 2 34
13 39 0 37 4 24 31 20 10 0 30 20 16 20 34
1 50 46 22 8 2 5 32 25 22 47 15 24 0 49
46 35 19 45 32 40 37 49 26 20 2 40 10 25 5
3 39 44 45 21 38 35 6 20 6 43 11 34 21 35
33 39 17 46 22 48 38 16 28 23 26 47 40 12 34
2 34 24 37 32 10 16 50 10 16 17 4 9 23 5
7 26 36 16 39 17 32 27 49 13 42 27 47 41 47
24 15 33 5 47 41 26 32 13 14 36 40 5 26 5
2 44 13 50 41 45 34 48 40 13 42 18 32 7 33
6 28 40 38 10 5 20 26 45 4 31 8 49 31 9
alex@wega:~/$ ./main < tests/01.t 
1 2 3 4 5 6 7 8 9 10 11 12 13 14 
alex@wega:~/$ cat tests/01.t 
3 5
37 5 24 41 8 10
37 45 50 9 44 37
1 12 43 17 38 47
alex@wega:~/$ ./main < tests/01.t 
-1
\end{alltt}
\pagebreak
