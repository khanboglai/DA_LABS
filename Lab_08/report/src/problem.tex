\CWHeader{Лабораторная работа \textnumero 8}

\CWProblem{
    Разработать жадный алгоритм решения задачи, определяемой своим вариантом. 
    Доказать его корректность, оценить скорость и объем затрачиваемой памяти.

    Реализовать программу на языке C или C++, соответсвующую построенному алгоритму. 
    Формат данных описан в варианте задания.


{\bfseries Вариант 4: Откорм бычков.} Бычкам дают пищевые добавки, чтобы ускорить их рост. 
Каждая добавка содержит некоторые из $N$ действующих веществ. Соотношения количеств веществ в добавках могут отличаться.

Воздействие добавки определяется как 

    $$c_1a_1 + c_2a_2 +· · ·+c_Na_N,$$ 

где $a_i$ — количество $i$-го вещества в добавке, 
$c_i$ – неизвестный коэффициент, связанный с веществом и не зависящий от добавки. Чтобы найти неизвестные коэффициенты $c_i$, 
Биолог может измерить воздействие любой добавки, использовав один её мешок. 
Известна цена мешка каждой из $M$ ($M <= N$) различных добавок. Нужно помочь Биологу подобрать самый дешевый наобор добавок, 
позволяющий найти коэффициенты $ci$. Возможно, 
соотношения веществ в добавках таковы, что определить коэффициенты нельзя.

{\bfseries Входные данные: } в первой строке текста – целые числа $M$ и $N$ ; в каждой из следующих $M$ строк записаны $N$ чисел, 
задающих соотношение количеств веществ в ней, а за ними – цена мешка добавки. 
Порядок веществ во всех описаниях добавок один и тот же, все числа – неотрицательные целые не больше $50$. 

{\bfseries Выходные данные: } $-1$ если определить коэффциенты невозможно, 
иначе набор добавок (и их номеров по порядоку во входных данных). Если вариантов несколько, вывести какой-либо из них. 

}
\pagebreak