\section{Выводы}
% {\itshape Здесь Вы пишите то, чему научились на лабораторной на самом деле, что узнали нового, где это может пригодиться и т.д. Мне важно, какие именно Вы сделали выводы из лабораторной.}

Выполнив восьмую лабораторную работу по курсу \enquote{Дискретный анализ}, 
я ознакомился с классическими задачами, которые решаются при помощи жадных алгоритмов, решил задачу для своего варианта, используя
жадный алгоритм.

Жадные алгоритмы делают выбор на основе локальной оптимальности и заранее определенных правил, 
что может привести к оптимальному решению в некоторых случаях, но не всегда. Динамическое программирование, в свою очередь, 
рассматривает все возможные варианты и гарантирует нахождение оптимального решения, что делает его более универсальным, 
но и более сложным в реализации.

Жадные алгоритмы могут эффективно решать определенные классы задач, обеспечивая быстрое получение решений. 
Благодаря своей простоте и скорости, жадные алгоритмы остаются важным инструментом в арсенале алгоритмических решений, 
особенно в задачах, где требуется быстрое приближение к оптимальному решению.

\pagebreak