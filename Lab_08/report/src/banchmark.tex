\section{Тест производительности}
% {\itshape Тут Вы описываете собственно тест производительности, сравнение Вашей реализации с уже существующими и т.д.}
Сравним алгоритм с наивным решение этой задачи.

Для тестирования возьмем наборы содержащие: 10 добавок, 15 добавок, 20 добавок, 25 добавок.

\begin{alltt}
alex@wega:~/$ ./becnh.sh 
Greedy: 0.008 ms
Naive: 1.554 ms
alex@wega:~/$ ./becnh.sh 
Greedy: 0.013 ms
Naive: 1.448 ms
alex@wega:~/$ ./becnh.sh 
Greedy: 0.025 ms
Naive: 1571.183 ms
alex@wega:~/$ ./becnh.sh 
Greedy: 0.032 ms
Naive: 2802.912 ms
\end{alltt}

Можем заметить, что жадный алгоритм значительно быстрее наивного. Это пороисходит потому, что сложность наивного алгоритма ---
$O(2^m*n^3)$ (перебор всех комбинаций $O(2^m)$, метод Гаусса $O(n^3)$), а сложность жадного алгоритма --- $O(m*n^2)$. 
Стоит отметить, что жадный алгоритм затрачивает меньше памяти, так как не хранит исходную систему уравнений.

\pagebreak
