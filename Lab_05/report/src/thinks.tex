\section{Выводы}
% {\itshape Здесь Вы пишите то, чему научились на лабораторной на самом деле, что узнали нового, где это может пригодиться и т.д. Мне важно, какие именно Вы сделали выводы из лабораторной.}

Выполнив пятую лабораторную работу по курсу \enquote{Дискретный анализ}, 
я изуил и реализовал алгоритм Укконена для построения суффиксного дерева за линейное время. 
Ознакомился с приложениями суффиксного дерева, а точнее со статистикой совпадений.

Статистика совпадений позволяет искать совпадающие подстроки в паттерне и тексте, 
при этом суффиксное дерево требует меньше памяти, потому что строиться по паттерну. 
Статистика совадений активно используется в базах данных для ускоренного поиска.

Суффиксные деревья лучше всего подходят для поиска нескольких шаблонов в одном тексте, 
так как остальные алгоритмы (Кнута-Морриса-Пратта, Бойера-Мура) не смогут также эффективно справиться с этой задачей.
Суффиксные деревья применяются в химии и биологии для работы с ДНК.
\pagebreak