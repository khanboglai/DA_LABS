\section{Тест производительности}
% {\itshape Тут Вы описываете собственно тест производительности, сравнение Вашей реализации с уже существующими и т.д.}

Реализованный алгоритм Укконена и алгоритм статистики совпадений сравним с наивным алгоритмом суффиксного дерева.

Будем тестировать на следующих строках и паттренах: (100, 2), (100, 50), (1000, 5), (1000, 100), (1000, 50), (10000, 1000). 
В скобках сначала длина строки, потом длина паттерна.

\begin{alltt}
[info] [2024-10-08 23:20:13] Running tests/01.t
Ukkonen+MS: 0.057 ms
Naive Suffix Tree: 2.394 ms
[info] [2024-10-08 23:21:19] Running tests/01.t
Ukkonen+MS: 0.045 ms
Naive Suffix Tree: 1.791 ms
[info] [2024-10-08 23:21:53] Running tests/01.t
Ukkonen+MS: 0.085 ms
Naive Suffix Tree: 123.346 ms
[info] [2024-10-08 23:22:34] Running tests/01.t
Ukkonen+MS: 0.198 ms
Naive Suffix Tree: 128.511 ms
[info] [2024-10-08 23:23:18] Running tests/01.t
Ukkonen+MS: 0.257 ms
Naive Suffix Tree: 121.333 ms
[info] [2024-10-08 23:24:11] Running tests/01.t
Ukkonen+MS: 0.851 ms
Naive Suffix Tree: 13044.072 ms
\end{alltt}

Как можно заметить, реализованный алгоритм работает быстрее, чем наивный. 
За счет суффиксных ссылок и прыжков по счётчику алгоритм является более производительным.
\pagebreak
